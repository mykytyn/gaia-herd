\documentclass{article}
\usepackage[utf8]{inputenc}
\usepackage{hyperref}
\usepackage{url}

\title{gaiasprint}
\author{dwm261 }
\date{March 2019}

\begin{document}

\maketitle

\section{Introduction}

Despite its long history, we still lack an analytical form for the Hertzprung-Russel (HR) diagram. 
One obstacle to constructing a forward model of the HR diagram, even for a single open cluster is the existence of apparent double and triple stars and interlopers. 
We will build on the work [WIDMARK, LEISTEDT, HOGG] to infer the HR diagram of a contaminated cluster. 
For the sprint, we want to be able to generate HR diagrams for open clusters from Gaia samples.

\section{Data}

Gaia cluster catalog: \url{http://vizier.u-strasbg.fr/viz-bin/VizieR-3?-source=J/A%2bA/618/A93/members}
For each cluster:\\
For each star (as much as possible of):
    \begin{itemize}
        \item Mags
        \item Colors
        \item Proper motions
        \item Angular location
        \item Parallax
    \end{itemize}

\section{Method}

We build on \url{https://arxiv.org/abs/1801.08547}, which identifies the probability of an apparent single star being a true binary or trinary assuming it's on the Main Sequence.
Their code is public here: \url{https://github.com/AxelWidmark/Multiple-stellar-systems-in-TGAS/blob/master/Model.py}.
We will use those probabilities to reconstruct the HR diagram of an open cluster.

\end{document}
